\documentclass[10pt,letterpaper,oneside,onecolumn]{article}
\usepackage[]{fontenc}
\usepackage[dvips]{epsfig}
\usepackage{fullpage}
\usepackage{html}

\pagestyle{plain}
\setcounter{secnumdepth}{5}
\setcounter{tocdepth}{5}

\title{A Brief Overview of Shimple}
\author{Navindra Umanee
(\htmladdnormallink{navindra@cs.mcgill.ca}{mailto:navindra@cs.mcgill.ca})}
\date{June 6, 2003}

\begin{document}
\maketitle

This document briefly describes Shimple, an SSA variant of Soot's
Jimple internal representation.  It assumes general knowledge of Soot,
Jimple and SSA form. You may wish to jump directly to the walk-through
section for a demonstration of why you might be interested in using
Shimple either by implementing SSA-based optimizations or by applying
them.

\section{Why Shimple?}

Static Single Assignment (SSA) form guarantees a single static
definition point for every variable used in a program, thereby
significantly simplifying as well as enabling certain analyses.

Shimple provides you with an IR in SSA form that is almost entirely
identical to Jimple except for the introduction of Phi nodes.  The
idea is that Shimple can be treated almost identically to Jimple with
the added benefits of SSA.  

For example, the additional variable splitting due to SSA form may
turn a previously flow-insensitive analysis into a flow-sensitive one
with little or no additional work.

\section{Hacking Overview}

The public API of Shimple is fully described in the
\htmladdnormallink{Soot API
documentation}{http://www.sable.mcgill.ca/soot/doc/}.  In particular,
in the \htmladdnormallink{soot.shimple
package}{http://www.sable.mcgill.ca/soot/doc/soot/shimple/package-summary.html},
the
\htmladdnormallink{Shimple}{http://www.sable.mcgill.ca/soot/doc/soot/shimple/Shimple.html}
class provides the Shimple grammar constructors and various utility
functions, the
\htmladdnormallink{ShimpleBody}{http://www.sable.mcgill.ca/soot/doc/soot/shimple/ShimpleBody.html}
class describes Shimple bodies and
\htmladdnormallink{PhiExpr}{http://www.sable.mcgill.ca/soot/doc/soot/shimple/PhiExpr.html}
provides the interface to Phi expressions.

Use/Definition and Definition/Use chains for Shimple bodies can be
obtained either from the accessor methods in ShimpleBody or with
\htmladdnormallink{ShimpleLocalDefs}{http://www.sable.mcgill.ca/soot/doc/soot/shimple/toolkits/scalar/ShimpleLocalDefs.html}
or
\htmladdnormallink{ShimpleLocalUses}{http://www.sable.mcgill.ca/soot/doc/soot/shimple/toolkits/scalar/ShimpleLocalUses.html}
in package soot.shimple.toolkits.scalar.

Available example analyses for Shimple currently include
ShimpleLocalDefs, SEvaluator and SConstantPropagatorAndFolder in
package
\htmladdnormallink{soot.shimple.toolkits.scalar}{http://www.sable.mcgill.ca/soot/doc/soot/shimple/toolkits/scalar/package-summary.html}.
Please consult the Soot source for details.

\section{Usage Options}

For a full description of the options and phases pertaining to
Shimple, please consult the
\htmladdnormallink{primary}{http://www.sable.mcgill.ca/soot/\#documentation}
Soot option and phase documentation.

\section{Command Line Walk Through}

For fun, you may wish to run Shimple from the command-line and study
its output.  Consider the following (compiled) Java code:

\begin{verbatim}
public class ShimpleTest
{
   public boolean doIt;

   public int test(){
     int i = 0;

     if (doIt)
         i = 1000;
     else
         i = 1000;

     return i;
   }
}
\end{verbatim}

\subsection{Producing Jimple}

If you produce Jimple with 
{\tt soot.Main -f jimple ShimpleTest}, you obtain the following code
for the {\tt test()} method:

\begin{verbatim}
        z0 = 0;
        if $z1 == 0 goto label0;

        i0 = 1000;
        goto label1;

     label0:
        i0 = 1000;

     label1:
        return i0;
\end{verbatim}

\subsection{Producing Shimple}

To produce Shimple instead, use {\tt soot.Main -f shimple ShimpleTest}:

\begin{verbatim}
        z0 = 0;
        if $z1 == 0 goto label0;

        i0 = 1000;
(0)     goto label1;

     label0:
(1)     i0_1 = 1000;

     label1:
        i0_2 = Phi(i0 #0, i0_1 #1);
        return i0_2;
\end{verbatim}

The difference between the Jimple and Shimple output is that the
latter guarantees unique local definition points in the program (for
scalars).  Notice here that the variable {\tt i0} has been split into
the three variables {\tt i0}, {\tt i0\_1}, and {\tt i0\_2}, each with a
unique definition point.

We have also introduced a Phi node.  You can read {\tt i0\_2 = Phi(i0
\#0, i0\_1 \#1)} as saying that {\tt i0\_2} will be assigned {\tt i0}
(that is, {\tt i0\_2 = i0}) if control flow comes from unit {\tt \#0},
{\em or} it will be assigned {\tt i0\_1} (that is, {\tt i0\_2 = i1})
if control flow comes from unit {\tt \#1}.

If you have a prejudice against variable names with underscores, you
may use {\tt soot.Main -f shimple -p shimple standard-local-names
ShimpleTest} instead so that Shimple applies the Local Name
Standardizer each time new locals are introduced.

Feel free to skip the following digression and move on to the next
subsection.

\subsubsection{A Digression on Shimple Pointers}

Because Soot represents the body of a method internally as a Unit
chain, we need to store the explicit pointers {\tt \#0} and {\tt \#1} to
keep track of the control flow predecessors of the Phi statements.

Shimple's internal implementation of PatchingChain attempts to move
and maintain these pointers in a manner that will be as transparent as
possible to the user.  For example, in the simplest case, if a
statement is appended to block:

\begin{verbatim}
     label0:
(1)     i0_1 = 1000;
\end{verbatim}

to obtain:

\begin{verbatim}
     label0:
        i0_1 = 1000;
(1)     new_stmt;
\end{verbatim}

Shimple will automatically move the {\tt \#1} pointer down to the new
statement since it is in the same basic block.

The intent is to provide maximum flexibility for code motion
optimizations as well as other transformations.  In this case, {\tt
i0\_1 = 1000} is free to move up or down the Unit chain as long as the
new location dominates the original CFG block it was in.

\subsection{Producing Jimple, Again}

Since we eventually have to get rid of those pesky Phi nodes, you
may wish to see what the code looks like after going from Jimple to
Shimple and back again to Jimple.  Do this with {\tt java soot.Main -f
jimple --via-shimple ShimpleTest}:

\begin{verbatim}
        if $z1 == 0 goto label0;

        i0_2 = 1000;
        goto label1;

     label0:
        i0_2 = 1000;

     label1:
        return i0_2;
\end{verbatim}

Happily, in this case, the Jimple produced looks exactly like the
original Jimple code.  As usual you may specify {\tt -p shimple
standard-local-names} if the underscores hurt your eyes; they are
otherwise quite harmless.

To understand what's really going on when Shimple eliminates Phi
nodes, you can tell it to eliminate them naively with {\tt soot.Main
-f jimple --via-shimple -p shimple phi-elim-opt:none ShimpleTest}:

\begin{verbatim}
        z0 = 0;
        if $z1 == 0 goto label0;

        i0 = 1000;
        i0_2 = i0;
        goto label1;

     label0:
        i0_1 = 1000;
        i0_2 = i0_1;

     label1:
        return i0_2;
\end{verbatim}

Now you can see that all Shimple did was to replace the Phi nodes
with equivalent copy statements.

\subsection{Applying Shimple Optimizations}

So, what good is Shimple?  

If you were paying attention, you may have noticed that in this
example, no matter which control flow path is taken, variable {\tt i}
is assigned a value of 1000 and is used by a single {\tt return}
statement.  In other words, {\tt i} is a constant and is otherwise
quite useless.  Obviously, this needs to be optimized away.

Let's try to apply Jimple's Constant Propagator and Folder.  In fact,
to be fair, let's try all the available  Jimple optimizations activated
with {\tt soot.Main -f jimple -p jop on ShimpleTest}:

\begin{verbatim}
        if $z1 == 0 goto label0;

        i0 = 1000;
        goto label1;

     label0:
        i0 = 1000;

     label1:
        return i0;
\end{verbatim}

As you can see in this case, the Jimple optimizations had trouble
tracking the control flow and failed to deduce that {\tt i} is a
constant.  Shimple, on the other hand, encodes control flow
information explicitly in the Phi nodes thereby allowing optimizations
to make use of the information.

Currently, the only optimization we have specifically implemented for
Shimple is a fairly naive and literal version of the constant
propagation algorithm sketched by the Cytron et el.  Let's apply it
with {\tt soot.Main -f jimple --via-shimple -p sop on ShimpleTest}:

\begin{verbatim}
        if $z1 == 0 goto label0;

        goto label0;

     label0:
        return 1000;
\end{verbatim}

Et voila, Shimple optimized out the {\tt i} variable completely and
replaced it with a constant.  What happened is that the optimization
propagated the constants to the Phi node and then noticed that the
Phi node was useless (because it made a selection from identical
values) and therefore trimmed it out.

To understand what is really going on, you can look at the output from
{\tt soot.Main -f shimple -p sop on} and {\tt soot.Main -f jimple
--via-shimple -p shimple phi-elim-opt:none -p sop on} on this and
other examples.

Although this example isn't sufficiently complex to be all that
interesting (a Jimple-based analysis could easily detect and handle
this particular case), once control flow gets more elaborate, the
SSA-based analysis will really start to win out.  

Perhaps you would like to experiment with slightly less simple 
examples and see how well the Shimple optimization fares:

\begin{verbatim}
public class ShimpleTest
{
   public boolean doIt, stopIt;

   public int test(){
     int i = 1000;
     int j = 1000;

     while(doIt != stopIt)
     {
        if (doIt)
           i = j;
        else
           j = i;
     }

     return i + j;
   }
}
\end{verbatim}

\section{Thanks and Credits}

Thanks and credits go alphabetically to Laurie Hendren, John
Jorgensen, Patrick Lam and Ondrej Lhotak for helping with the design
of Shimple and general implementation issues.

\section{Future Work}

Much more work on Shimple is planned as the project is likely to morph
into a Master's thesis.  Some thoughts currently include investigating
the various scalar SSA variants as well as heap, array and possibly
concurrent forms of SSA.  The Shimple architecture and implementation
will therefore evolve quite a bit internally, but as far as possible
we will try to maintain backwards-compatibility for the public
interfaces.

Suggestions, improvements and bug reports/fixes welcome!  Please send
these either to the Soot mailing list at
\htmladdnormallink{soot-list@sable.mcgill.ca}{mailto:soot-list@sable.mcgill.ca}, or directly to myself
at
\htmladdnormallink{navindra@cs.mcgill.ca}{mailto:navindra@cs.mcgill.ca}.

\subsection{Partial To-Do List}

\begin{itemize}
\item Implement more SSA-based analyses.
\item Add timers and profiling code.
\item Make internal analyses more useful and generic for external use.
\item Adopt an Strategy-type pattern for SSA builder modules, etc.
\item Implement a Shimple parser.
\item Provide an interface for CFG manipulations that intelligently 
updates Phi nodes.
\item Implement a Control Dependence Graph?  Any interest in that?
\end{itemize}

\subsection{Known Issues}

A vague description of a couple of known issues follows.  You may
ignore this section completely since regular usage of Shimple should
not be affected in general.

\subsubsection{Issue 1}

One issue is related to Phi nodes inserted at the beginning of try
blocks which are subsequently used by Phi nodes in the corresponding
handler block.  Fortunately, although the code produced looks strange
it is not incorrect.  Example:

\begin{verbatim}
    label1:
        i0_2 = Phi(i0 #0, i0_1 #1);
(2)     i1 = 4 / 0;
(3)     i0_3 = i0_2 / 0;
        i2 = i0_3;

     label2:
        goto label4;

     label3:
        $r0 := @caughtexception;
        i0_4 = Phi(i0_1 #1, i0 #0, i0_2 #2, i0_3 #3);

        catch java.lang.Exception from label1 to label2 with label3;
\end{verbatim}

The {\tt \#2} pointer of the second Phi node really should be pointing
directly at the first Phi node instead of at the statement following
it (since the latter may throw an exception and branch to the handler
block).  Fortunately, in these cases the second Phi node will always
be pointing directly at the predecessors of the first Phi node as well
({\tt \#0} and {\tt \#1} in this example), rendering the matter moot.
This glitch will be eliminated in a future release.

\subsubsection{Issue 2}

Another issue is related to the Shimple patching algorithm.  In the
rare case that control flow falls through from an if statement to a
try block and the if statement has a pointer to it:

\begin{verbatim}
       value = 1000;
(1)    if (whatever) goto label100;

label1:
       first_trap_statement;
       ...
       goto label100;
label2:
       $r0 := @caughtexception;
       i0 = Phi(value #1, ...);
\end{verbatim}

In the above, it may be desireable to move the \#1 pointer down if a
Unit happens to be inserted after the if statement.  Although Shimple
is smart enough to do this in most known cases, it currently misses
the one case where control flows from an if statement in a
non-exceptional context to an exceptional context.

This is not a big problem for most people unless an exotic code motion
algorithm (currently non-existent in Soot) attempts to move the
definition of {\tt value} below the if statement for some reason.

\end{document}


