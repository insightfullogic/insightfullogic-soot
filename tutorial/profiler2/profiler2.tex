\documentclass[12pt]{article}
\usepackage{fullpage}

\title{Using Soot to instrument a class file}
\author{Feng Qian}
\begin{document}
\maketitle

\section{Goals}
The purpose of this tutorial is to let you know:
\begin{enumerate}
\item how to inspect a class file by using Soot, and
\item how to profile a program by instrumenting the class file.
\end{enumerate}

\section{Illustration by examples}
I am going to show an example first. From the example, you can 
get some feelings about how to modify a class file using Soot.
Then I will explain internal representations of classes,
methods, and statements in Soot.

In this tutorial, I am making an example slightly different from the
example {\tt GotoInstrumenter} on the
web\footnote{http://www.sable.mcgill.ca/soot/tutorial/profiler/}.  You
should also check the old tutorial to learn how to add local variables
and fields, etc...

\vspace{.2in}
\noindent
{\Large Task :} count how many InvokeStatic instructions executed in a run of
a tiny benchmark {\Large TestInvoke.java :}
\begin{verbatim}
class TestInvoke {
  private static int calls=0;
  public static void main(String[] args) {
 
    for (int i=0; i<10; i++) {
      foo();
    }
 
    System.out.println("I made "+calls+" static calls");
  }
 
  private static void foo(){
    calls++;
    bar();
  }
 
  private static void bar(){
    calls++;
  }
}
\end{verbatim} 

\noindent
In order to record counters, I wrote a helper class called {\tt MyCounter}
{\Large MyCounter.java}
\begin{verbatim}
/* The counter class */
public class MyCounter {
  /* the counter, initialize to zero */
  private static int c = 0;
 
  /**
   * increases the counter by <pre>howmany</pre>
   * @param howmany, the increment of the counter.
   */
  public static synchronized void increase(int howmany) {
    c += howmany;
  }
 
  /**
   * reports the counter content.
   */
  public static synchronized void report() {
    System.err.println("counter : " + c);
  }
}
\end{verbatim}

\noindent
Now, I am creating a wrapper class to add a phase in Soot for
inserting profiling instructions, then call {\em
soot.Main.main()}. The main method of this driver class addes a
transformation phase named {\tt ``jtp.instrumenter''} to Soot's {\tt
``jtp''} pack. The {\tt PackManager} class the various Packs of phases of Soot.
When {\em MainDriver} makes a call
to {\em soot.Main.main}, Soot will know from {\em PackManager} that a new
phase was registered, and the {\tt internalTransform} method of
new phase will be called by Soot.  
{\Large MainDriver.java :}
\begin{verbatim}
     1  /* Usage: java MainDriver [soot-options] appClass
     2   */
     3
     4  /* import necessary soot packages */
     5  import soot.*;
     6
     7  public class MainDriver {
     8    public static void main(String[] args) {
     9
    10      /* check the arguments */
    11      if (args.length == 0) {
    12        System.err.println("Usage: java MainDriver [options] classname");
    13        System.exit(0);
    14      }
    15
    16      /* add a phase to transformer pack by call Pack.add */
    17      Pack jtp = PackManager.v().getPack("jtp");
    18      jtp.add(new Transform("jtp.instrumenter",
    19                            new InvokeStaticInstrumenter()));
    20
    21      /* Give control to Soot to process all options,
    22       * InvokeStaticInstrumenter.internalTransform will get called.
    23       */
    24      soot.Main.main(args);
    25    }
    26  }
\end{verbatim}

\noindent
The real implementation of instrumenter extends an abstract class
{\tt BodyTransformer}. It implements the {\tt internalTransform}
method which takes a method body (instructions) and some options. 
The main operations happen in this method. Depends on your command
line options, Soot builds a list of classes (which means a list of
methods also) and calls {\em InvokeStaticInstrumenter.internalTransform}
by passing in the body of each method. 
{\Large InvokeStaticInstrumenter.java :}
\begin{verbatim}
     1  /*
     2   * InvokeStaticInstrumenter inserts count instructions before
     3   * INVOKESTATIC bytecode in a program. The instrumented program will
     4   * report how many static invocations happen in a run.
     5   *
     6   * Goal:
     7   *   Insert counter instruction before static invocation instruction.
     8   *   Report counters before program's normal exit point.
     9   *
    10   * Approach:
    11   *   1. Create a counter class which has a counter field, and
    12   *      a reporting method.
    13   *   2. Take each method body, go through each instruction, and
    14   *      insert count instructions before INVOKESTATIC.
    15   *   3. Make a call of reporting method of the counter class.
    16   *
    17   * Things to learn from this example:
    18   *   1. How to use Soot to examine a Java class.
    19   *   2. How to insert profiling instructions in a class.
    20   */
    21
    22  /* InvokeStaticInstrumenter extends the abstract class BodyTransformer,
    23   * and implements <pre>internalTransform</pre> method.
    24   */
    25  import soot.*;
    26  import soot.jimple.*;
    27  import soot.util.*;
    28  import java.util.*;
    29
    30  public class InvokeStaticInstrumenter extends BodyTransformer{
    31
    32    /* some internal fields */
    33    static SootClass counterClass;
    34    static SootMethod increaseCounter, reportCounter;
    35
    36    static {
    37      counterClass    = Scene.v().loadClassAndSupport("MyCounter");
    38      increaseCounter = counterClass.getMethod("void increase(int)");
    39      reportCounter   = counterClass.getMethod("void report()");
    40    }
    41
    42    /* internalTransform goes through a method body and inserts
    43     * counter instructions before an INVOKESTATIC instruction
    44     */
    45    protected void internalTransform(Body body, String phase, Map options) {
    46      // body's method
    47      SootMethod method = body.getMethod();
    48
    49      // debugging
    50      System.out.println("instrumenting method : " + method.getSignature());
    51
    52      // get body's unit as a chain
    53      Chain units = body.getUnits();
    54
    55      // get a snapshot iterator of the unit since we are going to
    56      // mutate the chain when iterating over it.
    57      //
    58      Iterator stmtIt = units.snapshotIterator();
    59
    60      // typical while loop for iterating over each statement
    61      while (stmtIt.hasNext()) {
    62
    63        // cast back to a statement.
    64        Stmt stmt = (Stmt)stmtIt.next();
    65
    66        // there are many kinds of statements, here we are only
    67        // interested in statements containing InvokeStatic
    68        // NOTE: there are two kinds of statements may contain
    69        //       invoke expression: InvokeStmt, and AssignStmt
    70        if (!stmt.containsInvokeExpr()) {
    71          continue;
    72        }
    73
    74        // take out the invoke expression
    75        InvokeExpr expr = (InvokeExpr)stmt.getInvokeExpr();
    76
    77        // now skip non-static invocations
    78        if (! (expr instanceof StaticInvokeExpr)) {
    79          continue;
    80        }
    81
    82        // now we reach the real instruction
    83        // call Chain.insertBefore() to insert instructions
    84        //
    85        // 1. first, make a new invoke expression
    86        InvokeExpr incExpr= Jimple.v().newStaticInvokeExpr(increaseCounter,
    87                                                    IntConstant.v(1));
    88        // 2. then, make a invoke statement
    89        Stmt incStmt = Jimple.v().newInvokeStmt(incExpr);
    90
    91        // 3. insert new statement into the chain
    92        //    (we are mutating the unit chain).
    93        units.insertBefore(incStmt, stmt);
    94      }
    95
    96
    97      // Do not forget to insert instructions to report the counter
    98      // this only happens before the exit points of main method.
    99
   100      // 1. check if this is the main method by checking signature
   101      String signature = method.getSubSignature();
   102      boolean isMain = signature.equals("void main(java.lang.String[])");
   103
   104      // re-iterate the body to look for return statement
   105      if (isMain) {
   106        stmtIt = units.snapshotIterator();
   107
   108        while (stmtIt.hasNext()) {
   109          Stmt stmt = (Stmt)stmtIt.next();
   110
   111          // check if the instruction is a return with/without value
   112          if ((stmt instanceof ReturnStmt)
   113              ||(stmt instanceof ReturnVoidStmt)) {
   114            // 1. make invoke expression of MyCounter.report()
   115            InvokeExpr reportExpr= Jimple.v().newStaticInvokeExpr(reportCounter);
   116
   117            // 2. then, make a invoke statement
   118            Stmt reportStmt = Jimple.v().newInvokeStmt(reportExpr);
   119
   120            // 3. insert new statement into the chain
   121            //    (we are mutating the unit chain).
   122            units.insertBefore(reportStmt, stmt);
   123          }
   124        }
   125      }
   126    }
   127  }
\end{verbatim}

\noindent
Now, test the instrumenter, before instrumentation:
\begin{verbatim}
[cochin] [621tutorial] java TestInvoke
I made 20 static calls
\end{verbatim}

\noindent
Run the instrumenter:
\begin{verbatim}
[cochin] [621tutorial] java MainDriver TestInvoke
Soot started on Tue Feb 12 21:22:59 EST 2002
Transforming TestInvoke... instrumenting method : <TestInvoke: void <init>()>
instrumenting method : <TestInvoke: void main(java.lang.String[])>
instrumenting method : <TestInvoke: void foo()>
instrumenting method : <TestInvoke: void bar()>
instrumenting method : <TestInvoke: void <clinit>()>
 
Soot finished on Tue Feb 12 21:23:02 EST 2002
Soot has run for 0 min. 3 sec.
\end{verbatim}

\noindent
Run the benchmark again:
\begin{verbatim}
[cochin] [621tutorial] java TestInvoke
I made 20 static calls
counter : 20
\end{verbatim}

\noindent
Compare the JIMPLE code before and after instrumentation:

\noindent
{\tt BEFORE :}
\begin{verbatim}
     1  class TestInvoke extends java.lang.Object
     2  {
            ......
    14      public static void main(java.lang.String[] )
    15      {
             ......
    26       label0:
    27          staticinvoke <TestInvoke: void foo()>();
    28          i0 = i0 + 1;
                ......
    42          return;
    43      }
    44
    45      private static void foo()
    46      {
                ......
    52          staticinvoke <TestInvoke: void bar()>();
    53          return;
    54      }
    55
    56      private static void bar()
    57      {
                ......
    63          return;
    64      }
            ......
    71  }
\end{verbatim}

\noindent
{\tt AFTER:}
\begin{verbatim}
     1  class TestInvoke extends java.lang.Object
     2  {
            ......
    14      public static void main(java.lang.String[] )
    15      {
                ......
    26       label0:
    27          staticinvoke <MyCounter: void increase(int)>(1);
    28          staticinvoke <TestInvoke: void foo()>();
    29          i0 = i0 + 1;
                ......
    43          staticinvoke <MyCounter: void report()>();
    44          return;
    45      }
    46
    47      private static void foo()
    48      {
                ......
    54          staticinvoke <MyCounter: void increase(int)>(1);
    55          staticinvoke <TestInvoke: void bar()>();
    56          return;
    57      }
    58
    59      private static void bar()
    60      {
                ......
    66          return;
    67      }
            ......
    74  }
\end{verbatim}

\noindent
We see that a method call to {\tt MyCounter.increase(1)} was added before
each {\tt staticinvoke} instruction, and a call to {\tt MyCounter.report()}
was inserted before the {\tt return} instruction in {\tt main} method.

\section{More on soot tutorial web page}
{\tt http://www.sable.mcgill.ca/soot/tutorial/}

\end{document}
