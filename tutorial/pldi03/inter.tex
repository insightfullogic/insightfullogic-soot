\begin{slide}{Program and Cast}
\begin{description}
\item[ACT I ({\em Warming Up}):] \hspace{1in} \\
\begin{itemize}
\item Introduction and Soot Basics {\blue (Laurie)}
\item Intraprocedural Analysis in Soot {\blue (Patrick)}
\end{itemize}
\item[ACT II ({\em The Home Stretch}):] \hspace{1in} \\
\begin{itemize}
\item {\red Interprocedural Analyses and Call Graphs {\blue (Ondrej)}}
\item Attributes in Soot and Eclipse {\blue (Ondrej,Feng,Jennifer)}
\item Conclusion, Further Reading \& Homework {\blue (Laurie)}
\end{itemize}
\end{description}
\end{slide}

\begin{slide}{Interprocedural Outline}
\begin{itemize}
\item Soot's whole-program mode
\item Call graph
\item Points-to information (Spark)
\begin{itemize}
\item (Spark was my M.Sc. thesis)
\end{itemize}
\end{itemize}
\end{slide}

\begin{slide}{Soot's whole-program mode}
\begin{itemize}
\item Use \texttt{-w} switch for whole-program mode
\item Enables cg, wjtp, wjap packs
\item Whole-program information from these packs available to rest of Soot through Scene
\begin{itemize}
\item Call graph
\item Points-to information
\end{itemize}
\item Whole program analyzed; only application classes written out, not library classes
\item To also enable wjop, use \texttt{-W}
\begin{itemize}
\item Method inlining, static binding
\end{itemize}
\end{itemize}
\end{slide}

\begin{slide}{Soot Phases (Whole Program)}
\begin{description}
\item[cg] generates a call graph using CHA or more precise methods
\item[wjtp] performs user-defined whole-program transformations
\item[wjop] performs whole-program optimizations
\begin{itemize}
\item static inlining
\item static method binding
\end{itemize}
\item[wjap] generates annotations using whole-program analyses
\begin{itemize}
\item rectangular array analysis
\end{itemize}
\end{description}
\end{slide}

\newcommand{\rep}[1]{#1 #1 #1 #1}
\newcommand{\scenephase}[2]{%
\pstree[treemode=R]{\Tr{\psframebox{\vbox to6cm{\vspace{\fill}\hbox{\rotatebox{270}{#1}}\vspace{\fill}}}}}{#2}}
\newcommand{\jbphase}[2]{%
\pstree[treemode=L]{#2}{\rep{\Tr{\psframebox{\vbox to1cm{\vspace{\fill}\hbox{#1}\vspace{\fill}}}}}}}
\newcommand{\bodyphase}[2]{%
\pstree[treemode=R]{\Tr{\psframebox{\vbox to1cm{\vspace{\fill}\hbox{#1}\vspace{\fill}}}}}{#2}}

\begin{slide}{(A selection of) Soot phases}
\begin{center}
\psset{levelsep=*.3,treesep=.4}
\jbphase{jb}{\scenephase{cg}{\scenephase{wjtp}{\scenephase{wjop}{\scenephase{wjap}{%
\rep{\bodyphase{jtp}{\bodyphase{jop}{\bodyphase{jap}{\bodyphase{bb}{\bodyphase{tag}{}}}}}}}}}}}
\end{center}
\end{slide}




\newcommand{\comment}[1]{}


\begin{slide}{Call Graph}
\begin{itemize}
\item Collection of edges representing {\red all} method invocations known to Soot
\begin{itemize}
\item explicit method invocations
\item implicit invocations of static initializers
\item implicit calls of {\tt Thread.run()}
\item implicit calls of finalizers
\item implicit calls by {\tt AccessController}
\item \ldots
\end{itemize}
\item {\tt Filter} can be used to select specific kinds of edges
\end{itemize}
\end{slide}

\comment{
\begin{slide}{Updating Call Graph}
\begin{itemize}
\item {\tt addEdge( Edge )}
\item {\tt removeEdge( Edge )}
\item Call graph queries always give up-to-date information
\end{itemize}
\end{slide}
}


\begin{slide}{Call Graph Edge}
\begin{itemize}
\vspace*{-5mm}
\item Each {\tt Edge} contains
\begin{itemize}
\item Source method
\item Source statement (if applicable)
\item Target method
\item Kind of edge
\end{itemize}
\end{itemize}
\newcommand{\dott}[1]{\rnode{#1}{$\bullet$}}
\begin{tabular}{|c|c|c|c|}
\hline
source m. & source stmt. & target m. & kind\\
\hline
\dott{src} & \dott{stmt} & \dott{tgt} & VIRTUAL \\
\hline
\end{tabular}
\newcommand{\tab}{\hspace*{8mm}}
\vspace*{6mm}\\
\begin{minipage}{2in}
\tt
\rnode{foo}{foo()} \{\\
\tab \rnode{foos}{o.bar();}\\
\}
\end{minipage}
\begin{minipage}{2in}
\tt
\rnode{bar}{bar()} \{ \\
\tab /* */ \\
\}
\end{minipage}
\newcommand{\arrow}[2]{\nccurve[arrowsize=.3,angleA=270,angleB=45]{->}{#1}{#2}}
\arrow{src}{foo}\arrow{tgt}{bar}\arrow{stmt}{foos}\nccurve[arrowsize=.3,linestyle=dashed,angleB=180]{->}{foos}{bar}
\end{slide}

\begin{slide}{Edge Kinds}
{
\tiny\bf
\begin{verbatim}
// Due to explicit invokestatic instruction.
public static final int STATIC = 1;
// Due to explicit invokevirtual instruction.
public static final int VIRTUAL = 2;
// Due to explicit invokeinterface instruction.
public static final int INTERFACE = 3;
// Due to explicit invokespecial instruction.
public static final int SPECIAL = 4;
// Implicit call to static initializer.
public static final int CLINIT = 5;
// Implicit call to Thread.run() due to Thread.start() call.
public static final int THREAD = 6;
// Implicit call to Thread.exit().
public static final int EXIT = 7;
// Implicit call to non-trivial finalizer from constructor.
public static final int FINALIZE = 8;
// Implicit call to run() through AccessController.doPrivileged().
public static final int PRIVILEGED = 9;
\end{verbatim}
}
\end{slide}

\begin{slide}{Querying Call Graph}
\begin{description}
\item[\tt targetsOf(SootMethod)] iterator over edges with given source method
\item[\tt targetsOf(Unit)] iterator over edges with given source statement
\item[\tt callersOf(SootMethod)] iterator over edges with given target method
\begin{tabular}{|c|c|c|c|}\hline {\red src}&o.foo();&foo()&VIRTUAL\\\hline\end{tabular}
\begin{tabular}{|c|c|c|c|}\hline {\red src}&o.goo();&goo()&VIRTUAL\\\hline\end{tabular}
\begin{tabular}{|c|c|c|c|}\hline {\red src}&o.hoo();&hoo()&VIRTUAL\\\hline\end{tabular}
{\gray \begin{tabular}{|c|c|c|c|}\hline bar&o.foo();&foo()&VIRTUAL\\\hline\end{tabular}}
\end{description}
\end{slide}

\begin{slide}{Querying Call Graph}
\begin{description}
\item[\tt targetsOf(SootMethod)] iterator over edges with given source method
\item[\tt targetsOf(Unit)] iterator over edges with given source statement
\item[\tt callersOf(SootMethod)] iterator over edges with given target method
\begin{tabular}{|c|c|c|c|}\hline src&o.foo();&{\red foo()}&VIRTUAL\\\hline\end{tabular}
{\gray \begin{tabular}{|c|c|c|c|}\hline src&o.goo();&goo()&VIRTUAL\\\hline\end{tabular}}
{\gray \begin{tabular}{|c|c|c|c|}\hline src&o.hoo();&hoo()&VIRTUAL\\\hline\end{tabular}}
\begin{tabular}{|c|c|c|c|}\hline bar&o.foo();&{\red foo()}&VIRTUAL\\\hline\end{tabular}
\end{description}
\end{slide}

\begin{slide}{Adapters}
\begin{itemize}
\item Adapters make an iterator over edges into an iterator over
\begin{description}
\item[{\tt Sources}] source methods
\item[{\tt Units}] source statements
\item[{\tt Targets}] target methods
\end{description}
\end{itemize}
\newcommand{\src}[1]{\begin{tabular}{|c|}\hline {\red $src_#1$}\\\hline\end{tabular}}
\newcommand{\edge}[1]{\begin{tabular}{|c|c|c|c|}\hline {\red $src_#1$}&$stmt_#1$&$tgt_#1$&$kind_#1$\\\hline\end{tabular}}
\begin{minipage}{2.5in}
\edge{1}\\
\edge{2}\\
\edge{3}\\
\end{minipage}
\begin{minipage}{1in}
\src{1}\\
\src{2}\\
\src{3}\\
\end{minipage}
\end{slide}

\begin{slide}{Code Example}
\begin{verbatim}
void mayCall( SootMethod src,
        CallGraph cg ) {
   Iterator edges = cg.targetsOf(src);
   Iterator targets =
                  new Targets(edges);
   while( targets.hasNext() ) {
      SootMethod tgt =
         (SootMethod) targets.next();
      System.out.println( ""+
        src+" may call "+tgt );
   }
}
\end{verbatim}
\end{slide}

\begin{slide}{Transitive Targets}
\begin{itemize}
\item {\tt TransitiveTargets} class takes a {\tt CallGraph} and
optional {\tt Filter} to select edges
\end{itemize}
\begin{description}
\item {\tt iterator(SootMethod)} iterator over methods transitively
called from given method
\item {\tt iterator(Unit)} iterator over methods transitively
called from targets of given statement
\end{description}
\end{slide}

\comment{
\begin{slide}{Entry Points}
\begin{itemize}
\item Kept separate from call graph edges
\item {\tt EntryPoints} class returns default entry points
\begin{description}
\item[\tt application()] main method and main class static initializer
\item[\tt implicit()] methods implicitly called by VM
\item[\tt all()] all of the above
\end{description}
\end{itemize}
\end{slide}
}

\begin{slide}{Implementation Big Picture}
\newcommand{\cls}[2]{\rnode{#2}{\psframebox{\begin{tabular}{c}#1\end{tabular}}}}
\vspace*{-5mm}
\begin{psmatrix}[colsep=.2,rowsep=.4]
\cls{Entry Points}{ep}\\
&\cls{Scene}{scene}\\\ \\
\cls{Reachable\\Methods}{rm}&\cls{Call Graph\\Builder}{cgb}&\cls{Points-to}{pt}\\
\ \\
&&\cls{Naive}{naive} \cls{Spark}{spark}
\end{psmatrix}
\psset{arrowsize=.3}
\nccurve[angleA=270,angleB=175]{->}{ep}{scene}\mput*{\tiny methods}
\nccurve[angleA=185,angleB=90]{->}{scene}{rm}\mput*{\tiny methods}
\nccurve[angleA=270,angleB=265]{->}{rm}{cgb}\mput*{\tiny methods}
\nccurve[angleA=90,angleB=80]{->}{cgb}{rm}\mput*{\tiny edges}
\nccurve[angleA=90,angleB=135]{->}{cgb}{pt}\mput*{\tiny edges}
\nccurve[angleA=225,angleB=275]{->}{pt}{cgb}\mput*{\tiny receiver types}
\ncangles[angleA=90,angleB=270,arrowinset=0]{->}{naive}{pt}
\ncangles[angleA=90,angleB=270,arrowinset=0]{->}{spark}{pt}
\end{slide}

\comment{
\begin{slide}{Reachable Methods}
\begin{itemize}
\item {\tt ReachableMethods} class requires
\begin{itemize}
\item collection of entry points
\item call graph
\item optional edge filter
\end{itemize}
\item Updates list of reachable methods as edges added to call graph
\item Edge removals are ignored
\end{itemize}
\end{slide}

\begin{slide}{Queues}
\begin{itemize}
\item Used throughout Soot to communicate events
\item Objects added to queue can be read by multiple QueueReaders
\begin{itemize}
\item QueueReaders are like Iterators
\end{itemize}
\item In Call Graph, there are Queues for
\begin{itemize}
\item added edges
\item methods that become reachable
\end{itemize}
\end{itemize}
\end{slide}

\begin{slide}{Call Graph Builder}
\begin{itemize}
\item Analyzes methods that become reachable, adding edges
to call graph
\item Uses {\tt PointsToAnalysis} interface to resolve receiver
types for virtual calls
\begin{itemize}
\item {\tt DumbPointsToAnalysis} gives CHA
\item Spark gives RTA, VTA, more precise analyses
\end{itemize}
\end{itemize}
\end{slide}
}


\begin{slide}{Points-to analysis}
\begin{itemize}
\item Default points-to analysis assumes that any pointer can point to any object
\item Spark provides variations of context-insensitive subset-based
points-to analysis
\begin{itemize}
\item Work in progress on context-sensitive analyses
\end{itemize}
\end{itemize}
\end{slide}

\begin{slide}{Spark settings}
\begin{itemize}
\item \texttt{-p cg.spark on} turns on Spark
\begin{itemize}
\item Spark used for both call graph, and points-to information
\item Default setting is on-the-fly call graph, field-sensitive,
most efficient algorithm and data structures
\end{itemize}
\item \texttt{-p cg.spark vta} Spark as VTA
\item \texttt{-p cg.spark rta} Spark as RTA
\end{itemize}
\end{slide}

\begin{slide}{PointsToAnalysis interface}
\begin{description}
\item
[\texttt{\small reachingObjects(Local)}]
returns {\tt PointsToSet} of objects pointed to by a local variable\\
{\tt x = y}

\item
[\texttt{\small reachingObjects(SootField)}]
returns {\tt PointsToSet} of objects pointed to by a static field\\
{\tt x = C.f}

\item
[\texttt{\small reachingObjects(Local,SootField)}]
returns {\tt PointsToSet} of objects pointed to by given instance field
of the objects pointed to by local variable\\
{\tt x = y.f}
\end{description}
\sablefootnote{soot.PointsToAnalysis}
\end{slide}

\begin{slide}{PointsToSet interface}
\begin{description}
\item [\texttt{\small possibleTypes()}] returns a set of the possible types of the
objects in the points-to set
\item [\texttt{\small hasNonEmptyIntersection(PointsToSet)}] tells us whether
two points-to sets may overlap (whether the pointers may be aliased)
\end{description}
\sablefootnote{soot.PointsToSet}
\end{slide}

\begin{slide}{If I want to know...}
... the types of the receiver o in the call: \texttt{o.m(\ldots)}\\
\hspace{10mm}
\begin{alltt}
Local o;
PointsToAnalysis pa =
     Scene.v().getPointsToAnalysis();
PointsToSet ptset = 
     pa.reachingObjects( o );
java.util.Set types =
     ptset.possibleTypes()
\end{alltt}
\end{slide}

\begin{slide}{If I want to know...}
... whether \texttt{x} and \texttt{y} may be aliases in\\
\texttt{x.f = 5;\\ y.f = 6;\\ z = x.f;}\\
\hspace{10mm}
\begin{alltt}
Local x, y;
PointsToSet xset =
    pa.reachingObjects( x );
PointsToSet yset =
    pa.reachingObjects( y );
if(xset.hasNonEmptyIntersection(yset))
    // they're possibly aliased
\end{alltt}
\end{slide}

\begin{slide}{SideEffectTester interface}
Reports side-effects of any statement, including calls
\begin{description}
\item [\texttt{newMethod(SootMethod)}] tells the side-effect tester that
we are starting a new method
\item [\texttt{unitCanReadFrom(Unit,Value)}]
returns true if the Unit (statement) might read the Value
\item [\texttt{unitCanWriteTo(Unit,Value)}]
returns true if the Unit (statement) might write the Value
\end{description}
\sablefootnote{soot.SideEffectTester}
\end{slide}

\begin{slide}{Implementations of SideEffectTester}
\begin{description}
\item [\texttt{NaiveSideEffectTester}]\ \\
\begin{itemize}
\item is conservative
\item does not use call graph or points-to information
\item does not require whole-program mode
\end{itemize}
\item [\texttt{PASideEffectTester}]\ \\
\begin{itemize}
\item uses current call graph
\item uses current points-to information
\begin{itemize}
\item this may be naive points-to information
\end{itemize}
\end{itemize}
\end{description}
\sablefootnote{soot.jimple.NaiveSideEffectTester\\soot.jimple.toolkits.pointer.PASideEffectTester}
\end{slide}


