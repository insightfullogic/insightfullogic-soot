\begin{slide}{Flow Analysis in Soot}
\vspace*{0.2in}
\begin{itemize}
\item Flow analysis is key part of compiler framework
\item Soot has easy-to-use framework for intraprocedural flow analysis
\item Soot itself, and its flow analysis framework, are object-oriented.
\end{itemize}
\end{slide}

\begin{slide}{Four Steps to Flow Analysis}
\begin{enumerate}
\item Forward or backward?  Branched or not?

\item Decide what you are approximating.\\
What is the domain's confluence operator?

\item Write equation for each kind of IR statement.

\item State the starting approximation.
\end{enumerate}
\end{slide}

\begin{slide}{HOWTO: Soot Flow Analysis}
A checklist of your obligations:
\begin{enumerate}
\item Subclass \verb+*FlowAnalysis+
\item Implement abstraction: {\tt merge()}, {\tt copy()}
\item Implement flow function {\tt flowThrough()}
\item Implement initial values: {\tt newInitialFlow()} and {\tt
entryInitialFlow()}
\item Implement constructor \\ \quad (it must call {\tt doAnalysis()})
\end{enumerate}
\end{slide}

\begin{slide}{HOWTO: Soot Flow Analysis II}
Soot provides you with:
\begin{itemize}
\item impls of abstraction domains (flow sets)
\begin{itemize}
\item standard abstractions trivial to implement;
\end{itemize}
\item an implemented flow analysis namely,
\begin{itemize}
\item {\tt doAnalysis()} method:
executes intraprocedural analyses on a
CFG using a worklist algorithm.
\end{itemize}
\end{itemize}
\end{slide}

\begin{slide}{Flow Analysis Hierarchy}
\Tree [.AbstractFlowAnalysis [.FlowAnalysis Forward- Backward- ] [.BranchedFlowAnalysis Forward- ] ]

\end{slide}

\begin{slide}{Soot Flow Analyses}
\begin{center}
{\small 
\Tree [.AbstractFlowAnalysis 
[.FlowAnalysis [.Forward- {PRE analy's\\Avail. Expr.\\Array Bds} ] [.Backward- { PRE analy's \\ Liveness }  ] ] 
[.BranchedFlowAnalysis [.Forward- Casts Nullness ] ] 
]
}
\end{center}
\end{slide}


\begin{slide}{Outline: Soot Flow Analysis Examples}
Will describe how to implement a flow analysis in Soot and present examples:
\begin{itemize}
\item live locals
\item branched nullness testing
\end{itemize}
\end{slide}


\begin{slide}{Running Example 1: Live Variables}
\vspace*{-0.2in} 
A local variable {\tt v} is {\red live} at $s$ if there exists some
statement $s'$ using {\tt v} and a control-flow path from $s$ to $s'$ free
of definitions of {\tt v}.

\quad

\begin{center}
\input{lv-example.eepic}
\end{center}
\end{slide}

\begin{slide}{Steps to a Flow Analysis}
As we've seen before:
\begin{enumerate}
\item Subclass \verb+*FlowAnalysis+
\item Implement abstraction: {\tt merge()}, {\tt copy()}
\item Implement flow function {\tt flowThrough()}
\item Implement initial values: {\tt newInitialFlow()} and {\tt
entryInitialFlow()}
\item Implement constructor \\ \quad (it must call {\tt doAnalysis()})
\end{enumerate}
\end{slide}

\begin{slide}{Step 1: Forward or Backward?}

Live variables is a backward flow analysis, since flow
f$^{\mbox{\scriptsize n}}$ computes {\sf IN} sets from {\sf OUT} sets.

\qquad

In Soot, we subclass {\tt \red BackwardFlowAnalysis}.

\qquad

{\red \tt class LiveVariablesAnalysis \\ \qquad extends BackwardFlowAnalysis}
\end{slide}

\begin{slide}{Backward vs. Forward Analyses}
\vspace*{-0.24in}
A forward analysis computes {\sf OUT} from {\sf IN}:

\vspace*{-0.05in}
\begin{center}
\input{flow-fwd.eepic}
\end{center}

\vspace*{-0.05in}
A backward analysis computes {\sf IN} from {\sf OUT}:

\begin{center}
\input{flow-bkwd.eepic}
\end{center}

\end{slide}

\begin{slide}{Step 2: Abstraction domain}
\vspace*{-0.2in}

Domain for Live Variables: sets of {\tt Local}s

\qquad e.g. {\tt  \{x, y, z\}}

\begin{itemize}

\item Partial order is subset inclusion

\item Merge operator is union 

\end{itemize} 

In Soot, we use the provided {\tt ArraySparseSet} implementation of {\tt
FlowSet}.
\end{slide}

\begin{slide}{Implementing an Abstraction}
\vspace*{-0.2in}
Need to implement {\tt copy()}, {\tt merge()} methods:
\begin{center}
\input{copy-bkwd.eepic}
\end{center}
{\tt copy()} brings IN set to predecessor's OUT set.

\begin{center}
\input{merge-bkwd.eepic}
\end{center}
{\tt merge()} joins two IN sets to make an OUT set.
\end{slide}

\begin{slide}{More on Implementing an Abstraction}
Signatures:
{\small 
\[ \mbox{\tt \begin{tabular}{r@{}l}
                 void merge(&Object src1, Object src2,\\
                            &Object dest);\\
                 void copy(&Object src, Object dest);
\end{tabular}}\]
}

We delegate implementation to {\tt FlowSet}.
\end{slide}

\begin{slide}{Flow Sets and Soot}
\vspace*{-0.1in}
Using a {\tt FlowSet} is not mandatory, but helpful.

\quad

Impls: {\tt ToppedSet}, {\tt ArraySparseSet}, \\
\qquad \qquad {\tt ArrayPackedSet}

\quad

\begin{tabular}{ll}
\begin{minipage}{0.5\textwidth}
{\small \tt 
// $c = a \cap b$ \\
a.intersection(b, c); \\

// $d = \overline{c}$\\
c.complement(d);
}
\end{minipage} &
\begin{minipage}{0.3\textwidth}
{\small \tt
// $c = a \cup b$ \\
a.union(b,c);\\

// $d = d \cup \{ \mbox{v} \}$\\
d.add(v);
}
\end{minipage}

\end{tabular}

\quad

\sablefootnote{www.sable.mcgill.ca/soot/doc/soot/toolkits/scalar/FlowSet.html}

% why flow universes?
%  * efficient encoding 
%  * we sometimes want to go a \cup \not b.
\end{slide}

\begin{slide}{Digression: types of {\tt FlowSet}s}
\vspace*{-0.1in}
Which {\tt FlowSet} do you want?
\begin{itemize}
%\item {\tt ArraySparseSet}: represents as simple array; 
%can't be complemented, needs no initial universe
\item {\tt ArraySparseSet}: simple list

\qquad \begin{tabular}{|c|c|c|c|c|} \hline
foo & bar & z & \\ \hline
\end{tabular}

(simplest possible)

%% \item {\tt ArrayPackedSet}: represents as bitvector array;
%% can be complemented, needs initial universe

\item {\tt ArrayPackedSet}: bitvector w/ map

\qquad \begin{tabular}{|c|c|c|} \hline
00100101 & 10101111 & 10000000 \\ \hline
\end{tabular}

(can complement, need universe)

\item {\tt ToppedSet}: 

\qquad {\tt FlowSet \& {\tt isTop()} }

(adjoins a $\top$ to another {\tt FlowSet})

%$\top$ is above all other flowset elts (used in CSE).
\end{itemize}
\end{slide}

\begin{slide}{Step 2: {\tt copy()} for live variables}
\begin{verbatim}
protected void copy(Object src, 
                    Object dest) {
  FlowSet sourceSet = (FlowSet)src,
      destSet = (FlowSet) dest;

\end{verbatim}    
{\red\verb+  sourceSet.copy(destSet);+}
\begin{verbatim}
}
\end{verbatim}

\qquad

Use {\tt copy()} method from {\tt FlowSet}.
\end{slide}

\begin{slide}{Step 2: {\tt merge()} for live variables}
In live variables, a variable {\tt v} is live if there exists {\red any} path
from {\tt d} to {\tt p}, so we use {\red union}.

\qquad

Like {\tt copy()}, use {\tt FlowSet}'s {\tt union}:

\vspace*{0.05in}

\begin{verbatim}
  void merge(...) {
    // [cast Objects to FlowSets]
\end{verbatim}
{\red\verb+    src1Set.union(src2Set, destSet);+}\\
\verb+  }+

\vspace*{0.1in}

One might also use {\tt intersection()}, or implement a more exotic merge.

\end{slide}

\begin{slide}{Step 3: Flow equations}
\vspace*{-0.1in}

Goal: At a unit like {\tt x = y * z}:\\
\qquad \qquad {\red kill \quad } def {\tt x};\\
\qquad \qquad {\red gen $\- \- \- $ } uses {\tt y}, {\tt z}.

\vspace*{0.1in}

How? Implement this method:
\begin{verbatim}
    protected void flowThrough
                 (Object srcValue, 
                  Object u, 
                  Object destValue)
\end{verbatim}
\end{slide}

\begin{slide}{Step 3: Casting}
Soot's flow analysis framework is polymorphic.
Need to cast to do useful work.

$\quad$

Start by:
\begin{itemize}
\item casting {\tt srcValue}, {\tt destValue} to {\tt FlowSet}.

\item casting {\tt u} to {\tt Unit ut}.
\end{itemize}

In code:

\begin{verbatim}
   FlowSet src = (FlowSet)srcValue,
           dest = (FlowSet)destValue;
   Unit ut = (Unit)u;
\end{verbatim}
\end{slide}

\begin{slide}{Step 3: Copying}
Need to copy {\tt src} to {\tt dest} to allow manipulation.

\begin{center}
\input{copy-bkwd.eepic}
\end{center}

\begin{verbatim}
   src.copy (dest);
\end{verbatim}

$\quad$

Use {\tt FlowSet} methods.
\end{slide}

\begin{slide}{Step 3: Implementing {\tt flowThrough}}
Must decide what happens at each statement (in general, need to switch
on unit type):
\begin{center}
\scalebox{1.3}{
\input{flowthrough-bkwd.eepic}}
\end{center}
{\tt flowThrough} is the brains of a flow analysis.
\end{slide}

\begin{slide}{Step 3: {\tt flowThrough} for live locals}
\vspace*{0.1in}
A local variable {\tt v} is {\red live} at $s$ if there exists some
statement $s'$ containing a use of {\tt v},
and a control-flow path from $s$ to $s'$ free
of def'ns of {\tt v}.

$\quad$

Don't care about the type of unit we're analyzing: Soot provides
abstract accessors to values used and defined in a unit.

\end{slide}

\begin{slide}{Step 3: Implementing {\tt flowThrough}: removing kills}
\begin{verbatim}
// Take out kill set:
//  for each local v def'd in 
//  this unit, remove v from dest
Iterator boxIt = ut.getDefBoxes()
                     .iterator();
while( boxIt.hasNext() ) {
    ValueBox box = 
         (ValueBox) boxIt.next();
    Value value = box.getValue();
    if( value instanceof Local )
        dest.remove( value );
}
\end{verbatim}
\end{slide}

\begin{slide}{Step 3: Implementing {\tt flowThrough}: adding gens}
\vspace*{-0.2in}
\begin{verbatim}
// Add gen set
//  for each local v used in 
//  this unit, add v to dest
boxIt = ut.getUseBoxes().iterator();
while (boxIt.hasNext()) {
    ValueBox box = 
         (ValueBox) boxIt.next();
    Value value = box.getValue();
    if (value instanceof Local)
        dest.add(value);
}
\end{verbatim}

\vspace*{0.05in}

N.B. our analysis is generic, not restricted to Jimple.
\end{slide}

\begin{slide}{Step 4: Initial values}
\vspace*{-0.15in}
$\bullet$ Soundly initialize IN, OUT sets prior to analysis.

\vspace*{-0.1in}
\begin{itemize}
\item Create initial sets\\
{\tt Object newInitialFlow()}\\
\qquad {\tt \{return new ArraySparseSet();\} }

\item Create initial sets for exit nodes\\
{\tt Object entryInitialFlow() \\ 
\qquad \{return new ArraySparseSet();\} }

\end{itemize}

Want conservative initial value at exit nodes, optimistic
value at all other nodes.
\end{slide}

\begin{slide}{Step 5: Implement constructor}
\begin{verbatim}
LiveVariablesAnalysis(UnitGraph g)
{
    super(g);

    doAnalysis();
    
}
\end{verbatim}
Causes the flow sets to be computed, using Soot's flow
analysis engine.

\quad

In other analyses, we precompute values.
\end{slide}

% branching flow analysis

\begin{slide}{Running Example 2: Branched Nullness}
\vspace*{-0.2in} 
A local variable {\tt v} is {\red non-null} at $s$ if all
control-flow paths reaching $s$ result in {\tt v} being
assigned a value different from {\tt null}.

\begin{center}
\input{br-example.eepic}
\end{center}
\end{slide}

\begin{slide}{HOWTO: Soot Flow Analysis}
Again, here's what to do:
\begin{enumerate}
\item Subclass \verb+*FlowAnalysis+
\item Implement abstraction: {\tt merge()}, {\tt copy()}
\item Implement flow function {\tt flowThrough()}
\item Implement initial values: {\tt newInitialFlow()} and {\tt
entryInitialFlow()}
\item Implement constructor \\ \quad (it must call {\tt doAnalysis()})
\end{enumerate}
\end{slide}

\begin{slide}{Step 1: Forward or Backward?}

Nullness is a branched forward flow analysis, since flow
f$^{\mbox{\scriptsize n}}$ computes {\sf OUT} sets from {\sf IN} sets, 
sensitive to branches

\qquad

In Soot, we subclass {\tt \red ForwardBranchedFlowAnalysis}.

\qquad

{\small \red \tt class NullnessAnalysis \\ 
           \qquad extends ForwardBranchedFlowAnalysis \{ }
\end{slide}

\begin{slide}{Step 2: Abstraction domain}
\vspace*{-0.2in}

Domain: sets of {\tt Local}s known to be non-null\\
Partial order is subset inclusion.

\quad
\vspace*{-0.05in}

(More complicated abstractions possible$^{*}$ for this problem; e.g. 
$\perp, \top, \mbox{\tt null}, \mbox{non-{\tt null}}$ per-local.)

\quad
\vspace*{-0.05in}

Again use {\tt ArraySparseSet} to implement:
{\small 
\[ \mbox{\tt \begin{tabular}{r@{}l}
                 void merge(&Object in1, Object in2,\\
                            &Object out);\\
                 void copy(&Object src, Object dest);
\end{tabular}}\]
}

\sablefootnote{$^*$ BranchedRefVarsAnalysis}
\end{slide}

\begin{slide}{Implementing an Abstraction}
\vspace*{-0.2in}
For a forward analysis, {\tt copy} and {\tt merge} mean:
\begin{center}
\input{copy-fwd.eepic}
\end{center}
{\tt copy()} brings OUT set to predecessor's IN set.

\begin{center}
\input{merge-fwd.eepic}
\end{center}
{\tt merge()} joins two OUT sets to make an IN set.
\end{slide}

\begin{slide}{Step 2: {\tt copy()} for nullness}
Same as for live locals.

\begin{verbatim}
protected void copy(Object src, 
                    Object dest) {
  FlowSet sourceSet = (FlowSet)src,
      destSet = (FlowSet) dest;

\end{verbatim}    
{\red\verb+  sourceSet.copy(destSet);+}
\begin{verbatim}
}
\end{verbatim}

\qquad

Use {\tt copy()} method from {\tt FlowSet}.
\end{slide}

\begin{slide}{Step 2: {\tt merge()} for nullness}
In branched nullness, a variable {\tt v} is non-null if it is non-null
on all paths from {\tt start} to {\tt s}, so we use intersection.

\qquad

Like {\tt copy()}, use {\tt FlowSet} method -- here, {\tt intersection()}:

\vspace*{0.05in}

\begin{verbatim}
  void merge(...) {
    // [cast Objects to FlowSets]
\end{verbatim}
{\red\verb+    srcSet1.intersection(srcSet2,+\\
     \verb+                         destSet);+}\\
\verb+  }+

\end{slide}

\newcounter{plamfoo}

\overlays{6}{
\begin{slide}{Step 3: Flow equations}
\vspace*{-0.1in}
We do the following things in our flow function:

\begin{itemstep}
\item Create clone of src set.
\item Remove kill set (defined {\tt Local}s).\\
\qquad \qquad {\tt y} in {\tt y = y.next};
\item Add gen set.\\
\qquad \qquad {\tt x} in {\tt x.foo()};
\item Handle copy statements.
\item Copy to branch and fallthrough lists.
\item Patch sets for {\tt if} statements.
\end{itemstep}
\end{slide}}

\begin{slide}{Step 4: Initial values}
\vspace*{-0.1in}
Initialize IN, OUT sets.
\begin{itemize}

\item Create initial sets ($\top$ from constr.)\\
{\tt Object newInitialFlow() \{\\
\qquad \tt \{ return fullSet.clone(); \} }

\vspace*{0.1in}

\item Create entry sets ({\tt emptySet} from constr.)\\
{\tt Object entryInitialFlow()}\\
\qquad {\tt \{ return emptySet.clone(); \} }
\end{itemize}

(To be created in constructor!)
\end{slide}

\begin{slide}{Step 5: Constructor: Prologue}
\vspace*{-0.1in}
Create auxiliary objects.
\vspace*{0.05in}
\begin{verbatim}
public NullnessAnalysis(UnitGraph g)
{
   super(g);

   unitToGenerateSet = new HashMap();
   Body b = g.getBody();
   List refLocals = new LinkedList();
\end{verbatim}
\end{slide}

\begin{slide}{Step 5: Constructor: Finding All Locals}
\vspace*{-0.2in}
Create flowsets, finding all locals in body:

\vspace*{-0.05in}
\begin{verbatim}
emptySet = new ArraySparseSet();
fullSet = new ArraySparseSet();

Iterator localIt = 
  b.getLocals().iterator();

while (localIt.hasNext()) {
    Local l = (Local)localIt.next();
    if (isRefOrArrayType(l.getType()))
        fullSet.add(l);
}
\end{verbatim}
\end{slide}

\begin{slide}{Step 5: Creating gen sets}
Precompute, for each statement, which locals become non-null after execution of that stmt.

\begin{itemize}
\item {\tt x} gets non-null value:\\
{\tt x = *}, where {\tt *} is {\tt NewExpr}, {\tt ThisRef}, etc.
\item successful use of {\tt x}:\\
{\tt x.f}, {\tt x.m()}, {\tt entermonitor x}, etc.
\end{itemize}
\end{slide}

\begin{slide}{Step 5: Constructor: Doing work}
\vspace*{-0.1in}
Don't forget to call {\tt doAnalysis()}!

\vspace*{0.05in}
\begin{verbatim}
        ...

        doAnalysis();
    }
}
\end{verbatim}
\end{slide}

% do we really need 'copy'?  I wonder how much time is spent on that.
% if our flow set implementation has immutability, then 'copy' can be
% simple pointer assignment.

% why 'Object' rather than 'FlowSet'?  We sometimes don't want to
% wrap a HashSet into an adapting FlowSet (design choice)